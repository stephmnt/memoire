\documentclass{article}
\usepackage[utf8]{inputenc}
\usepackage[T1]{fontenc}
\usepackage[french]{babel}

\author{Stéphane Manet}
\date{\today} 
 
\title{% Un peu de bordel pour un sous-titre
    \begin{minipage}\linewidth
        \centering\bfseries
        Apprendre à innover
        \vskip3pt
        \large Pré-projet de mémoire pour le MERFA du CNAM.
    \end{minipage}
} 
 
\begin{document}
\maketitle % Titre du document
\tableofcontents

% Stuff pour moi
% https://archivesic.ccsd.cnrs.fr/sic_01560460/document
% https://www.scienceshumaines.com/qu-est-ce-que-l-esprit-creatif_fr_26404.html
% https://www.scienceshumaines.com/le-developpement-de-la-creativite_fr_5213.html
% https://www.meshs.fr/page/apprendre_a_creer
% http://www.toupie.org/Dictionnaire/Innovation.htm

\pagebreak

\section*{Introduction}

Après 5 ans en tant que formateur dans le numérique et dans un contexte social, j'ai souhaité enrichir mon parcours du titre Responsable de projet de formation au CNAM afin d'amorcer une reconversion partielle. 
Par ailleurs, j'ai été \emph{community manager} il y a 10 ans, au moment où le métier se créait tout juste, sans fiche de poste ou diplôme associé. 

En avançant dans mon projet, et surtout au regard des cours qui m'ont accompagné et permis de le formaliser, je souhaite pouvoir travailler avec des métiers émergents, en cours de développement et, en parallèle, poursuivre mes études jusqu'à la recherche. Je souhaite pouvoir concentrer cet intérêt dans les sciences de l'éducation.

Afin de trouver une adéquation dans tous ces domaines d'intérêt, j'aimerais faire une recherche dans le monde des incubateurs et des \emph{start-ups} sur le thème de l'innovation.

\section{Définition du sujet}

En regardant l'étymologie de l'innovation, on a \emph{in} et \emph{novare} : « rendre nouveau ». Dans l'émission de radio Cause à Effet (93.1 FM), Étienne Klein dit :

\begin{quote}
\textit{La rhétorique de l'innovation qu'on entend déclinée à peu près partout dans les entreprises et aussi dans les laboratoires de recherche, en gros, c'est une rhétorique de la sauvegarde. C'est-à-dire que sans qu'on l'entende toujours de façon explicite, ce qu'on explique c'est qu'il faut innover pour qu'on empêche que le monde se défasse.}\footnote{Cause à Effet, du 20 février 2018 avec Étienne Klein, animé par Stéphane Manet sur la radio locale Cause Commune 93.1 FM en Île-de-France, où Sandrine de Magondeaux, chroniqueuse, le fait réagir au livre sous la direction de Gilles Amado : \emph{La créativité au travail}, Érès, 2017.}
\end{quote} 

Dans un contexte où le monde managérial et de la gestion avance aussi vite que les entreprises ont besoin de rester compétitives dans un environnement économique avec des mutations très fortes, il existe un matériau de recherche nouveau (et peut-être parfois pas si nouveau) qui peut nous éclairer sur ce qu'il se produit en termes de travail et d'apprentissage.

\subsection{Intérêt en science de l'éducation}

\og \textit{La pédagogie est un art qui doit s'appuyer sur des connaissances scientifiques actualisées\footnote{Olivier Houdé (2016), \emph{Pour une pédagogie scientifique : allers-retours du labo à l'école}. Administration \& Éducation, n\degre 152, 9-15.} }.\fg En touchant à la question de la créativité, on touche à la fois au plus proche de ce qui fait l'apprenance, et ce qui fait la transformation. Comment apprendre à \og donner l'existence, tirer du néant, réaliser quelque chose qui n'existait pas encore \fg{}, si apprendre, c'est déjà transformer son rapport au réel ?

\subsection{Intérêt en analyse du travail}

L'injonction à l'innovation\footnote{Etienne Klein, \emph{Sauvons le progrès, dialogue avec Denis Lafay}, l’Aube, 2017.}, les réunions qui encadrent le \og design thinking \fg{} ou \og théorie C-K \fg{} dans un temps et un contexte donné biaisent l'idée même de créativité.
Elles créent des contextes de travail nouveau (esprit start-up, les heures \og 20\% Google \fg{}) et qui tendent à se généraliser. 
Face à ces phénomènes nouveaux, la recherche doit pouvoir apporter des réponses et convoquer les réponses qu'elle a déjà pu apporter sur les enjeux du travail et les risques psychosociaux.

\section{Problématique et objectifs de recherche}

\subsection{État de la recherche}

Si j'ai pu trouver beaucoup de matière chez les psychologues américains, c'est surtout dans les sciences de gestion que j'ai trouvé des sources francophones récentes.

La question de la créativité a également été abordée en tant que moyen\footnote{Gilles Amado, Jean-Philippe Bouilloud, Dominique Lhuilier, Anne-Lise Ulmann, \textit{La créativité au travail}, Érès, 2017}, sous l'angle du travail, je souhaite maintenant me concentrer sur la créativité en tant que but, et en tant que prescription.

\subsection{Postulat de départ}

Mon postulat de départ est de définir l'innovation comme une interaction humaine et cette interaction s'apprend comme une posture. Il existe donc un \og savoir innover \fg{}, et ce savoir peut se décliner en compétence. 
Je vais également regarder comment naissent les idées dans une organisation de travail afin de déterminer s'il est possible de les reconnaitre au sein de l'activité, plutôt que de chercher à les provoquer. 

\section{Démarche méthodologique}

C'est peut-être une innovation en soi d'interroger l'innovation au travers de la didactique professionnelle.

\subsection{Cadre théorique}

L'approche par la didactique professionnelle permettrait d'intégrer et de reconnaitre le concept d'idée au sein même de l'activité plutôt que d'en faire une prescription. Elle apporterait un regard contemporain en lien avec les enjeux des nouveaux dispositifs, notamment l'AFEST mais aussi les modalités comme le blended learning.

\subsection{Méthodes d'analyse}

Je m'inscris dans la recherche fondamentale de terrain \footnote{Yves Clot, \emph{La recherche fondamentale de terrain : une troisième voie}, Éducation Permanente n\degre 177, 2008}. Après avoir suivi les unités d'enseignement FAD 111 et 114, je ressens le besoin de créer mes propres outils d'analyse sur des terrains expérimentaux. À cet égard, je souhaite en particulier travailler sur des métiers mal définis, dont les fiches métiers et les référentiels n'existent pas encore, bref, des métiers \og qui n'existent pas \fg{}.

\subsection{Contexte}

Idéalement, je souhaite pouvoir travailler dans une entreprise type incubateur à compter de la rentrée prochaine afin d'avoir un terrain, d'une part, et un financement du MERFA via un contrat de professionnalisation, d'autre part. 

\section{Bibliographie envisagée}

Voici pêle-mêle quelques ouvrages que j'envisage de lire ou que j'ai eu l'occasion de lire mais que je souhaiterais recontextualiser. Cette bibliographie est à actualiser avec l'aide de ma directrice de recherche. 

\subsection{Clinique du travail}

\begin{itemize}

\item Sous la direction de Gilles Amado, Jean-Philippe Bouilloud, Dominique Lhuilier, Anne-Lise Ulmann, \textit{La créativité au travail}, Érès, 2017
\item Yves Clot, \textit{Le travail sans l’homme}, La Découverte, 2008.

\end{itemize}

\subsection{Psychologie}

\begin{itemize}

\item Todd Lubart, \textit{Psychologie de la créativité}, Armand Colin, 2003
\item Ronald A. Finke, Thomas B. Ward et Steven M. Smith, \textit{Creative Cognition: Theory, research, and applications}, MIT Press, 1992.
\item Mark A. Runco, \textit{Creativity: Theories and themes: Research, development, and practice}, Academic Press, 2007.
\item Howard Gardner, \textit{Les Formes de la créativité}, Odile Jacob, 2001.
\item Sebastian Dieguez, \textit{Maux d’artistes. Ce que cachent les œuvres}, Belin, 2010.
\item Jacques Cottraux, \textit{À chacun sa créativité. Einstein, Mozart, Picasso… et nous}, Odile Jacob, 2010.

\end{itemize}

\subsection{Économie et Gestion}

\begin{itemize}

\item Gilles Garel et Elmar Mock, \textit{La Fabrique de l'innovation}, Dunod, 2016.
\item Tom Kelley, \textit{The Ten Faces Of Innovation}, Doubleday, 2008.

\end{itemize}

\subsection{Philosophie}

\begin{itemize}

\item Francis Bacon, \textit{Essais de morale et de politique}, 1625.
\item Étienne Klein, \textit{D'où viennent les idées scientifiques}, Broché, 2013.
\item Étienne Klein, \textit{Sauvons le progrès, dialogue avec Denis Lafay}, l’Aube, 2017.
\item Mahamadé Savadogo, \textit{Théorie de la création}, Broché, 2016.	

\end{itemize}

\subsection{En sciences de l’éducation}

\begin{itemize}

\item Olivier Reboul, \textit{Qu'est-ce qu'apprendre ?}, PUF, 2010.

\end{itemize}

\subsection{Divers}

\begin{itemize}

\item Todd Lubart et Chantal Pacteau, \textit{Le développement de la créativité}, Sciences Humaines, n\degre 164, octobre 2005.
\item Entretien de Nicolas Sadirac avec D. Iweins et F. Vairet, Les Échos, 29/09/2017.

\end{itemize}

\section{Perspectives}

\subsection{Limites de la recherche}

Dans le cadre du projet de mémoire, je m'interrogerai sur ce qui déclenche la créativité dans une entreprise ayant besoin d'innovation économique en démontrant qu'il existe un « savoir innover ». Il sera sans doute utile de poursuivre ces travaux par exemple sur ce qui déclenche une idée, seul ou en groupe, différence entre idée et créativité etc. Ce domaine ouvre de nombreuses perspectives.  

\subsection{Conclusion}

Dans ce contexte de recherche, je pense avoir trouvé l'adéquation entre les différents domaines avec lesquels je souhaite travailler. Je m'intéresse aux sciences de l'éducation, à la philosophie des sciences de l'éducation, et je touche le cœur d'un concept d'apprentissage complexe.

Je m'intéresse à l'épistémologie et je souhaite pouvoir développer mes propres outils d'analyses, et \emph{a fortiori} sur des métiers émergents, qui ne sont pas encore reconnus.

Je souhaite pouvoir faire avancer la recherche dans un univers professionnel avec des enjeux forts, nécessitant une approche syncrétique des sciences, et enfin pouvoir travailler dans ce domaine, et notamment en proposant d'accompagner ces travaux dans des dispositifs d'AFEST.

\end{document}