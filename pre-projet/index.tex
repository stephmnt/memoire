\documentclass{article}
\usepackage[utf8]{inputenc} % un package
\usepackage[T1]{fontenc}      % un second package
\usepackage[francais]{babel}  % un troisième package

\author{Stéphane Manet}
\title{Apprendre à innover}
\date{\today} 
 
\begin{document}
\maketitle % Titre du document
\tableofcontents

\section*{Introduction}

Après 5 ans en tant que formateur dans le numérique et dans un contexte social, j’ai souhaité enrichir mon parcours du titre Responsable de projet de formation au CNAM, afin d'amorcer une reconversion partielle. 
Par ailleurs, j'ai été \textit{community manager} il y a 10 ans, au moment où le métier se créait tout juste, sans fiche de poste ou diplôme associé. 

En avançant dans mon projet, et surtout au regard des cours qui m'ont accompagné et permis de le formaliser, je souhaite pouvoir travailler avec des métiers émergents, en cours de développement et, en parallèle, poursuivre mes études jusqu'à la recherche. Je souhaite pouvoir concentrer cet intérêt dans les sciences de l'éducation.

Afin de trouver une adéquation dans tous ces domaines d'intérêt, j'aimerais faire une recherche dans le monde des incubateurs et des \textit{start-ups} sur le thème de l'innovation.

\section{Définition du sujet}

En regardant l'étymologie de l'innovation, on a \textit{in} et \textit{novare} : « rendre nouveau ». Dans l'émission de radio Cause à Effet que j'anime sur la fréquence 93.1 en Île-de-France, Étienne Klein dit :

\begin{quote}
\textit{La rhétorique de l'innovation qu'on entend déclinée à peu près partout dans les entreprises et aussi dans les laboratoires de recherche, en gros, c'est une rhétorique de la sauvegarde. C'est-à-dire que sans qu'on l'entende toujours de façon explicite, ce qu'on explique c'est qu'il faut innover pour qu'on empêche que le monde se défasse.}\footnote{Cause à Effet, du 20 février 2018 avec Etienne Klein, animé par Stéphane Manet sur la radio locale île-de-France Cause Commune 93.1, où Sandine Magondeaux, chroniqueuse, le fait réagir au livre sous la direction de Gilles Amado, \textit{La créativité au travail}, Érès, 2017.}
\end{quote} 

\subsection{Intérêt en science de l'éducation}

SI les sciences de l'éducation ont déjà beaucoup d'avance dans le domaine des processus d'apprentissage, leur application dans le contexte du management de l'innovation est beaucoup plus rare. Le monde managérial et de la gestion avance aussi vite que les besoins des entreprises ont besoin de rester compétitives dans un environnement économique avec des mutations très fortes, au risque parfois de réinventer des concepts 

\subsection{Intérêt en analyse du travail}

L'injonction à l'innovation\footnote{Etienne Klein, \textit{Sauvons le progrès, dialogue avec Denis Lafay}, l’Aube, 2017.}, les réunions qui encadrent le « design thinking » ou « théorie C-K » dans un temps et un contexte donné biaisent l'idée même de créativité.
Elles créent des contextes de travail nouveau (esprit statut-up, babyfoot dans la salle de repos, les heures « 20\% Google ») et qui tendent à se généraliser. 
Face à ces phénomènes nouveaux, la recherche doit pouvoir apporter des réponses et au travers de ses réponses, convoquer les réponses que la recherche a déjà pu apporter sur les enjeux du travail, les risques psychosociaux, etc.

\section{Problématique et objectifs de recherche}

\subsection{État de la recherche}

Si on trouve beaucoup de matière chez les psychologues américains, c'est surtout dans les sciences de gestion que l'on trouve des sources francophones récentes. 

\section{Postulats de départ, objectifs et questions de recherche}

Mon postulat de départ est de définir l'innovation comme une interaction humaine et cette interaction s'apprend comme une posture. Il existe donc un « savoir innover », et ce savoir peut se décliner en compétence. 
Je vais également regarder comment naissent les idées dans une organisation de travail afin de déterminer s'il est possible de les reconnaitre au sein de l'activité, plutôt que de chercher à les provoquer. 

\section{Démarche méthodologique}

C'est peut-être une innovation en soi d'interroger l'innovation au travers de la didactique professionnelles. 

\subsection{Cadre théorique}

Une approche par la didactique professionnelle.

\subsection{Notions opératoires et modèles d'analyse}

Il convient de faire une recherche fondamentale (Yves Clot, la recherche fondamentale de terrain, Education Permanente).

\section{Problématique et objectifs de recherche}

\subsection{2.1. Etat de la recherche}

Si on trouve beaucoup de matière chez les psychologues américains, c'est surtout dans les sciences de gestion que l'on trouve des sources francophones récentes. 

\subsection{2.2. Postulats de départ, objectifs et questions de recherche}

Mon postulat de départ est de définir l'innovation comme une interaction humaine et cette interaction s'apprend comme une posture. Il existe donc un « savoir innover », et ce savoir peut se décliner en compétence. 
Je vais également regarder comment naissent les idées dans une organisation de travail afin de déterminer si l'est possible de les reconnaitre au sein de l'activité, plutôt que de chercher à les provoquer. 

\section{Démarche méthodologique}

C'est peut-être une innovation en soi d'interroger l'innovation au travers de la didactique professionnelles. 

\subsection{Cadre théorique}

Une approche par la didactique professionnelle.

\subsection{Notions opératoires et modèles d'analyse}

Il convient de faire une recherche fondamentale (Yves Clot, la recherche fondamentale de terrain, Éducation Permanente).

\section{Bibliographie}

\subsection{Clinique du travail}

\begin{itemize}

\item Sous la direction de Gilles Amado, Jean-Philipe Bouilloud, Dominique Lhuilier, Anne-Lise Ulmann, \textit{La créativité au travail}, Érès, 2017
\item Yves Clot, \textit{Le travail sans l’homme}, La Découverte, 2008.

\end{itemize}

\subsection{Psychologie}

\begin{itemize}

\item Todd Lubart, \textit{Psychologie de la créativité}, Armand Colin, 2003
\item Ronald A. Finke, Thomas B. Ward et Steven M. Smith, \textit{Creative Cognition: Theory, research, and applications}, MIT Press, 1992.
\item Mark A. Runco, Cr\textit{eativity: Theories and themes: Research, development, and practice}, Academic Press, 2007.
\item Howard Gardner, \textit{Les Formes de la créativité}, Odile Jacob, 2001.
\item Sebastian Dieguez, \textit{Maux d’artistes. Ce que cachent les œuvres}, Belin, 2010.
\item Jacques Cottraux, \textit{À chacun sa créativité. Einstein, Mozart, Picasso… et nous}, Odile Jacob, 2010.

\end{itemize}

\subsection{Economie et Gestion}

\begin{itemize}

\item Gilles Garel et Elmar Mock, \textit{La Fabrique de l'innovation}, Dunod, 2016.
\item Tom Kelley, T\textit{he Ten Faces Of Innovation}, Doubleday, 2008.

\end{itemize}

\subsection{Philosophie}

\begin{itemize}

\item Francis Bacon, \textit{Essais de morale et de politique}, 1625.
\item Étienne Klein, \textit{D'où viennent les idées scientifiques}, Broché, 2013.
\item Étienne Klein, Sa\textit{uvons le progrès, dialogue avec Denis Lafay}, l’Aube, 2017.
\item Mahamadé Savadogo, T\textit{héorie de la création}, Broché, 2016.	

\end{itemize}

\subsection{En sciences de l’éducation}

\begin{itemize}

\item Bourgeois, Reboul

\end{itemize}

\subsection{Divers}

\begin{itemize}

\item Todd Lubart et Chantal Pacteau, \textit{Le développement de la créativité}, Sciences Humaines, n 164, octobre 2005.
\item Entretien de Nicolas Sadirac avec D. Iweins et F. Vairet, Les Échos, 29/09/2017.

\end{itemize}

\section{Conclusion : perspectives et limites de la recherche}

Dans le cadre du projet de mémoire, je m'interrogerai sur ce qui déclenche la créativité dans une entreprise ayant besoin d'innovation économique en démontrant qu'il existe un « savoir innover ». Le projet   

\end{document}