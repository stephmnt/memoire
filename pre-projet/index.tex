\documentclass{article}
\usepackage[utf8]{inputenc}
% \usepackage{librebaskerville} % coming soon
\usepackage[T1]{fontenc}
\usepackage[french]{babel} % no rage plz

\author{Stéphane Manet}
\date{\today} 
 
\title{% Un peu de bordel pour un sous-titre
    \begin{minipage}\linewidth
        \centering\bfseries
        Rendre les actions d'accompagnement à l'innovation compatibles avec les dispositifs AFEST
        \vskip3pt
        \large Apprendre à innover : une approche par la didactique professionnelle
    \end{minipage}
} 
 
\begin{document}
\maketitle % Titre du document
\tableofcontents

% Stuff pour moi
% https://archivesic.ccsd.cnrs.fr/sic_01560460/document
% https://www.scienceshumaines.com/qu-est-ce-que-l-esprit-creatif_fr_26404.html
% https://www.scienceshumaines.com/le-developpement-de-la-creativite_fr_5213.html
% https://www.meshs.fr/page/apprendre_a_creer
% http://www.toupie.org/Dictionnaire/Innovation.htm

\pagebreak

\section*{Introduction}

Après 5 ans en tant que formateur dans le numérique et dans un contexte social, j'ai souhaité enrichir mon parcours du titre Responsable de projet de formation au CNAM afin d'amorcer une reconversion partielle. 
Par ailleurs, j'ai été \emph{community manager} il y a 10 ans, au moment où le métier se créait tout juste, sans fiche-métier ou diplôme associé. 

En avançant dans mon projet, et surtout au regard des cours qui m'ont accompagné et permis de le formaliser, je souhaite pouvoir travailler avec des métiers émergents, en cours de développement et, en parallèle, poursuivre mes études jusqu'à la recherche. Parmi les disciplines étudiées, ce sont les théories de l'apprentissage qui m'intéressent le plus.

Je souhaite par ailleurs travailler dans un contexte économique émergeant, le monde des incubateurs m'attire malgré le recul critique qu'il convient d'y porter\footnote{ou peut-être justement à cause de...}. 

Ainsi, afin de trouver une adéquation dans tous ces domaines d'intérêt, j'aimerais faire une recherche dans le monde des \emph{start-ups}, sur le thème de l'innovation. \footnote{voir section \ref{terrain} page \pageref{terrain}}

\section{Définition du sujet}

En regardant l'étymologie de l'innovation, on a \emph{in} et \emph{novare} : « rendre nouveau ». Dans l'émission de radio Cause à Effet (93.1 FM), Étienne Klein dit :

\begin{quote}
\textit{La rhétorique de l'innovation qu'on entend déclinée à peu près partout dans les entreprises et aussi dans les laboratoires de recherche, en gros, c'est une rhétorique de la sauvegarde. C'est-à-dire que sans qu'on l'entende toujours de façon explicite, ce qu'on explique c'est qu'il faut innover pour qu'on empêche que le monde se défasse.}\footnote{Cause à Effet, du 20 février 2018 avec Étienne Klein, animé par Stéphane Manet sur la radio locale Cause Commune 93.1 FM en Île-de-France, où Sandrine de Magondeaux, chroniqueuse, le fait réagir au livre sous la direction de Gilles Amado : \emph{La créativité au travail}, Érès, 2017.}
\end{quote} 

L'innovation n'est pas tout à fait la créativité. L'innovation est généralement associée à l'économie, au maintient de la compétitivité\footnote{Vicent Bontems, \textit{Que peut-on espérer de l'innovation ?}, Lettre de l'I-tésé n\degres 26, CEA, automne 2015}. L'innovation apparaît alors comme une injonction qui me permet d'aborder l'angle par les sciences de l'éducation tout autant que la clinique du travail.

Dans un contexte où le monde managérial et de la gestion avance aussi vite que les entreprises ont besoin de rester compétitives dans un environnement économique avec des mutations très fortes, il existe un matériau de recherche nouveau (et peut-être parfois pas si nouveau) qui peut nous éclairer sur ce qu'il se produit en termes de travail et d'apprentissage.

\subsection{Intérêt en clinique du travail}

% \footnote{Etienne Klein, \emph{Sauvons le progrès, dialogue avec Denis Lafay}, l’Aube, 2017.} // à replacer ?

L'injonction à l'innovation, les réunions qui encadrent le \og design thinking \fg{} ou \og théorie C-K \fg{} dans un temps et un contexte donné biaisent l'idée même de créativité.
Elles créent des contextes de travail nouveau (esprit start-up, les heures \og 20\% Google \fg{}) et qui tendent à se généraliser. 
Face à ces phénomènes nouveaux, la recherche doit pouvoir apporter des réponses et réinvestir les réponses qu'elle a déjà pu apporter sur les enjeux du travail et de la qualité de vie.

\subsection{Intérêt en science de l'éducation}

\og \textit{La pédagogie est un art qui doit s'appuyer sur des connaissances scientifiques actualisées\footnote{Olivier Houdé (2016), \emph{Pour une pédagogie scientifique : allers-retours du labo à l'école}. Administration \& Éducation, n\degres 152, 9-15.} }\fg{}. En touchant à la question de la créativité, on touche à la fois au plus proche de ce qui fait l'apprenance, et ce qui fait la transformation, comment apprendre à \og donner l'existence\fg{}, à \og tirer du néant \fg{}, à réaliser quelque chose qui n'existait pas encore, si apprendre c'est déjà transformer son rapport au réel ?\label{neant}

\section{Problématique et objectifs de recherche}

\subsection{L'objet}

Afin d'adapter l'objet du mémoire à ce domaine foisonnant, j'ai décidé de le resserrer autour d'un enjeu beaucoup plus contemporain :

\begin{quote}\textbf{Rendre les actions d'accompagnement à l'innovation  compatibles avec les dispositifs d'Action de Formation En Situation de Travail.}
\end{quote}

\subsection{Postulat de départ}

Mon postulat de départ est de définir l'innovation comme provenant d'une interaction humaine et cette interaction s'apprend comme une posture. Il existe donc un \og savoir innover \fg{}, et ce savoir peut se décliner en compétence. 

S'il se décline en compétence, il doit pouvoir rentrer dans un dispositif AFEST et se rendre éligible aux financements qui vont avec.

\subsection{État de la recherche}

Si j'ai pu trouver beaucoup de matière chez les psychologues américains, c'est surtout dans les sciences de gestion que j'ai trouvé des sources francophones récentes.

La question de la créativité a également été abordée en tant que moyen, notamment au travers du livre \textit{La créativité au travail}\footnote{Gilles Amado, Jean-Philippe Bouilloud, Dominique Lhuilier, Anne-Lise Ulmann, \textit{La créativité au travail}, Érès, 2017}, sous l'angle du travail, je souhaite maintenant m'y concentrer en tant que but, et en tant que prescription afin de voir qu'elles stratégies sont mises en œuvre pour la développer.

Un autre point important de la recherche que je dois pouvoir aborder, porte sur les actions de formation en situation de travail. Les AFEST ne sont pas encore rentrées dans les décrets au moment où j'écris ces lignes mais des expérimentations ont été menées préalablement sur lesquelles je compte m'appuyer. 

\section{Démarche méthodologique}

C'est peut-être une innovation en soi d'interroger l'innovation au travers de la didactique professionnelle.

\subsection{Direction de recherche}

Je souhaite que ce mémoire puisse être encadré par Capucine Bremond, qui m'a par ailleurs déjà aidé à la rédaction de ce pré-projet. Le regard particulier qu'elle porte sur la recherche scientifique, sur le sujet traité et la position qu'elle prend au sein de l'institution fond d'elle un acteur évident de ce projet, aussi je souhaite que ce rôle puisse être officialisé. 

\subsection{Cadre théorique}

L'approche par la didactique professionnelle permettrait d'intégrer et de reconnaître le concept d'idée au sein même de l'activité plutôt que d'en faire une prescription. Elle apporterait un regard contemporain en lien avec les enjeux des nouveaux dispositifs, notamment l'AFEST mais aussi les modalités comme le blended learning.

\subsection{Méthodes d'analyse}

Je m'inscris dans la recherche fondamentale de terrain \footnote{Yves Clot, \emph{La recherche fondamentale de terrain : une troisième voie}, Éducation Permanente n\degres 177, 2008}. Après avoir suivi les unités d'enseignement FAD 111 et 114\footnote{Respectivement : \og Analyse du travail et ingénierie de formation \fg{} et \og Développement des compétences en situation de travail \fg{}}, je ressens le besoin de créer mes propres outils d'analyse sur des terrains expérimentaux. À cet égard, je souhaite en particulier travailler sur des métiers mal définis, dont les fiches métiers et les référentiels n'existent pas encore, bref, des métiers qui \og n'existent pas \fg{}.

\subsection{Contexte}

Je compte travailler dans un incubateur à compter de la rentrée prochaine afin d'avoir un terrain, d'une part, et un financement du MERFA\footnote{Master Européen de Recherche en Formation des Adultes} via un contrat de professionnalisation, d'autre part.\label{terrain} 

\section{Bibliographie envisagée}

Voici pêle-mêle quelques ouvrages que j'envisage de lire ou que j'ai eu l'occasion de lire mais que je souhaiterais recontextualiser. Ils sont éclectiques afin d'être complémentaires aux ouvrages qui seront proposés dans le cadre du MERFA.

\subsection{Clinique du travail}

\begin{itemize}

\item Sous la direction de Gilles Amado, Jean-Philippe Bouilloud, Dominique Lhuilier, Anne-Lise Ulmann, \textit{La créativité au travail}, Érès, 2017
\item Yves Clot, \textit{Le travail sans l’homme}, La Découverte, 2008.

\end{itemize}

\subsection{Psychologie}

\begin{itemize}

\item Todd Lubart, \textit{Psychologie de la créativité}, Armand Colin, 2003
\item Ronald A. Finke, Thomas B. Ward et Steven M. Smith, \textit{Creative Cognition: Theory, research, and applications}, MIT Press, 1992.
\item Mark A. Runco, \textit{Creativity: Theories and themes: Research, development, and practice}, Academic Press, 2007.
\item Howard Gardner, \textit{Les Formes de la créativité}, Odile Jacob, 2001.
\item Sebastian Dieguez, \textit{Maux d’artistes. Ce que cachent les œuvres}, Belin, 2010.
\item Jacques Cottraux, \textit{À chacun sa créativité. Einstein, Mozart, Picasso… et nous}, Odile Jacob, 2010.

\end{itemize}

\subsection{Économie et Gestion}

\begin{itemize}

\item Gilles Garel et Elmar Mock, \textit{La Fabrique de l'innovation}, Dunod, 2016.
\item Tom Kelley, \textit{The Ten Faces Of Innovation}, Doubleday, 2008.

\end{itemize}

\subsection{Philosophie}

\begin{itemize}

\item Francis Bacon, \textit{Essais de morale et de politique}, 1625.
\item Étienne Klein, \textit{D'où viennent les idées scientifiques}, Broché, 2013.
\item Étienne Klein, \textit{Sauvons le progrès, dialogue avec Denis Lafay}, l’Aube, 2017.
\item Mahamadé Savadogo, \textit{Théorie de la création}, Broché, 2016.	

\end{itemize}

\subsection{En sciences de l’éducation}

\begin{itemize}

\item Olivier Reboul, \textit{Qu'est-ce qu'apprendre ?}, PUF, 2010.

\end{itemize}

\subsection{Divers}

\begin{itemize}

\item Todd Lubart et Chantal Pacteau, \textit{Le développement de la créativité}, Sciences Humaines, n\degres 164, octobre 2005.
\item Entretien de Nicolas Sadirac avec D. Iweins et F. Vairet, Les Échos, 29/09/2017.

\end{itemize}

\section{Perspectives}

\subsection{Limites de la recherche}

Ce domaine ouvre de nombreuses perspectives qui sont tout autant de portes dont chacun pourra se saisir. 

Comment naissent les idées dans une organisation de travail ? Quel en est le déclencheur ? Est-il possible de les reconnaître au sein de l'activité ? Que signifie \emph{réellement} apprendre à créer ? \footnote{voir section \ref{neant} page \pageref{neant}}

\subsection{Conclusion}

Dans ce contexte de recherche, je pense avoir trouvé l'adéquation entre les différents domaines avec lesquels je souhaite travailler. Je m'intéresse aux sciences de l'éducation, à la philosophie des sciences de l'éducation, et je touche le cœur d'un concept d'apprentissage complexe.

Je m'intéresse à l'épistémologie et je souhaite pouvoir développer mes propres outils d'analyses, et \emph{a fortiori} sur des métiers émergents, qui ne sont pas encore reconnus.

Je souhaite pouvoir faire avancer la recherche dans un univers professionnel avec des enjeux forts, nécessitant une approche syncrétique des sciences, et enfin pouvoir travailler dans ce domaine, et notamment en proposant d'accompagner ces travaux dans des dispositifs d'AFEST.

\end{document}
